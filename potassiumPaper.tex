\documentclass[letter,12pt]{article}

% Packages
\usepackage{hyperref}
\usepackage{physics}
\usepackage{amsmath}
\usepackage{cite}
\usepackage[margin=0.75in]{geometry}
% End Packages

% Shortcuts

\newcommand{\schr}{Schr\"odinger}

% End Shortcuts
\begin{document}
\title{A Review of Energy Splitting of Alkali Atoms in a Magnetic Field}
\author{Karl Ahrendsen}
\maketitle{}

\section{Introduction}
In Atomic, Molecular,
and Optical Physics, a topic of interest is the collisions between
basic 
objects, such as atoms, molecules, and elementary particles.
Our research group is focused on the study of spin-polarized electrons. 
An electron beam can be easily created and collided
with objects of interest(CITATION OF ELECTRON GUN WORK),
but this is not the most simplified interaction
which can occur, as electrons also have spin, which can distinguish one 
electron from another. We are interested in obtaining a source which
outputs electrons of only one polarization, referred to as a spin
polarized electron source.

Our research group has created a source of spin polarized
electrons using spin-exchange interactions with the alkali metal
rubidium.(CITATIONS, CITATIONS)  Our current method of obtaining 
polarized electrons involves first polarizing a rubidium vapor
with circularly polarized light. Then an unpolarized electron beam
is passed along the same path of the light 
through the polarized vapor. Spin exchange interactions
transfer the spin of the rubidium vapor to the electrons, and the 
result is a polarized electron beam. This approach makes intuitive
sense.

Another research group, Swensens, obtains the polarized electrons
in a different manner~\cite{swenson}. They pump their alkali metal, 
potassium, with
\emph{linearly} polarized light, perpendicular to the direction
of propagation of the electrons. This also must be done in the
presence of a magnetic field. The result is the same, a collection
of spin polarized alkali metal atoms, which could be used to create
a spin polarized electron beam. The goal of this paper is to thoroughly
explain how both of these approaches achieve the same result, as well 
as to examine the feasibility of performing this experiment in our
lab group.

This goal will  be accomplished by first explaining the basics 
of atomic physics for pumping alkali atoms, including notation, energy
levels, angular momentum, and selection rules. After this we
will examine how our description of the energy changes in the
presence of a magnetic
field. Once these basics of pumping have been established we will
examine the specific case of transverse optical pumping. 

\section{Basics of Energy Levels of Alkali Atoms}
The process of pumping a gas is based upon the transition
of an atom from some ground state to an excited state. 
It is therefore very important that one has a firm 
understanding of the quantum mechanics involved in this 
process. For this reason, I begin with a review of the energy
levels of an atom. For the sake of simplicity, I will complete
this discussion using
Hydrogen as the example atom, but the equations may be 
easily adjusted to be appropriate for an alkali atom.
These modifications will be discussed in Sections \ref{energyLevels}
and \ref{angularMomentum}
	\subsection{No Magnetic Field}
    We begin describing the energy levels in the absence of 
    a magnetic field. This consists of a slow buildup 
    from the Schr\"odinger equation starting with the radial
    equation and then adding in the orbital part. Intrinsic 
    angular momentum is introduced here before moving on to
    the selection rules which governs acceptable transitions
    between states. This is intended to be a swift tour through
	many main results of Quantum Mechanics relevant to 
	Optical pumping. If the reader is not familiar with the 
	concepts, any elementary quantum mechanics textbook
	such as Shankar, could be used to alleviate confusion.
		\subsubsection{Energy Levels}\label{energyLevels}
		The energy levels for hydrogen are found by solving the
		{\schr} equation with $V=-e^2/r$. 

		\begin{equation}
		  H\psi = E\psi
		\end{equation}
		where the Hamiltonian $H$ is 
		\begin{equation}
			H=-\frac{d^2}{dr^2}-\frac{2m}{\hbar}\Big[-\frac{e^2}{r} + \frac{l(l+1)\hbar^2}{2mr^2}\Big]
		\end{equation}
		
		Since this is a spherically
		symmetric potential, the Hamiltonian and angular momentum
		operators commute and we can split the wavefuntion into a
		radial part and an angular part. 
		\begin{equation}
			\psi(r,\theta,\phi)=R(r)Y(\theta,\phi)
		\end{equation}

		By examining only the radial part, we find that the energies
		are quantized and correspond to the values:
		\begin{equation}
			E_n=\frac{-me^4}{2\hbar^2n^2},\quad n=1,2,3,...
		\end{equation}

		The result from quantum mechanics that the energy levels 
		of the hydrogen atom are 
		quantized is of central importance to optical pumping. 
		Each energy level from the equation above 
		corresponds to a certain configuration of the 
		atom, or a state. The atom can transistion 
        between these states
		only if a special set of rules is fulfilled. 
        For our description so far, the only requirement
        to allow a transition between states
        is that additional energy, usually in the 
        form of electromagnetic radiation, 
        equal to the difference
        between the energy levels is put into the system. 
        More of these restrictions will appear as we
        expand our description of the hydrogen atom.
        In general, these restrictions 
        are called selection rules and will be further
        discussed in 
        Section \ref{selectionRules}.

        In order to more conveniently describe 
        this process of transitioning from one state to another
        we will introduce Dirac Bra-Ket notation. In 
        Bra-Ket notation a state is identified by 
        either a ``bra" $\bra{n}$
        or a ``ket" $\ket{n}$. Here, the n represents the energy
		level of the state. In our simplistic description thus far, 
		this is all we need to describe the state.
		As we add in more detailed descriptions of 
		the state, we will add in variables to completely 
		describe the state.

		The transition from one state to another in optical pumping
		will occur because of laser light incident on the electron.
		To account for this additional energy added into the system 
		we must augment
		our previous Hamiltonian. We call the Hamiltonian without 
		the energy from the light the unperturbed Hamiltonian and 
		give it the variable name $H_o$.
		\begin{equation}
			H_o=-\frac{d^2}{dr^2}-\frac{2m}{\hbar}\Big[-\frac{e^2}{r} + \frac{l(l+1)\hbar^2}{2mr^2}\Big]
		\end{equation}
		The energy from the light is called a pertubation and we
		can write its contribution to the atom as 
		\begin{equation}
			H_1=\frac{e}{2mc} \vec{A_o} \cdotp \vec{P}
		\end{equation}

		In the above expression, $e$ , $m$, and $p$ represent the charge, mass,
		and momentum of the electron, respectively. The variable $c$ represents
		the speed of light and $A$ represents the vector potential. 
		Here we are leaving out the derivation of this term and
		its time dependence because they will not be directly 
		relevant to optical pumping. Additionally, we have taken
		the electric dipole approximation. The interested reader should
		consult other texts for further information about the 
		pertubation and the approximation we make here. 

		Before continuing on to a discussion of angular momentum, 
		we will begin our discussion of how the alkali metals differ
		from hydrogen. The alkali metals are unique because they have
		a completely filled inner shells, or 

		Before further discussing this importance, the deviation of the alkali
		metals energy levels from the above hydrogen energies will be
		discussed. 
			The alkali atoms will have a similar energy spectrum to
		Hydrogen because of 

		\subsubsection{Angular Momentum}
        \label{angularMomentum}
		\subsubsection{Selection Rules}\label{selectionRules}

	\subsection{Weak Magnetic Field}
		\subsubsection{Energy Levels}
		\subsubsection{Coupling}
		\subsubsection{Selection Rules}

	\subsection{Strong Magnetic Field}

\section{Transverse Optical Pumping}

	\subsection{Required Setup}

	\subsection{Allowed Transitions}

	\subsection{Pumping}

\section{Conclusion}

the available values of the mixing element $\theta_{13}$.
\begin{equation}
	U=
	\begin{pmatrix}
		c_{12}c_{13} & s_{12}c_{13} & s_{13}e^{-i\delta}\\
		-s_{12}c_{23} -c_{12}s_{23}s_{13}e^{i\delta} & c_{12}c_{23} -s_{12}s_{23}s_{13}e^{i\delta} & s_{23}c_{13}\\
		s_{12}s_{23} -c_{12}c_{23}s_{13}e^{i\delta} & -c_{12}s_{23} -s_{12}c_{23}s_{13}e^{i\delta} & c_{23}c_{13}\\
	\end{pmatrix}
\end{equation}

\begin{equation}
	\begin{aligned}
		P(\nu_e \rightarrow \nu_\mu) = \sin[2](\theta_{23})\sin[2](2\theta_{13})
		\sin[2](\dfrac{1.27\Delta m^2L}{E})
	\end{aligned}
\end{equation}

\bibliographystyle{plain}
\bibliography{bibPotassium}

\end{document}
