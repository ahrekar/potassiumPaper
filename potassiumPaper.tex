\documentclass[letter,12pt]{article}

\usepackage{hyperref}
\usepackage{physics}
\usepackage{amsmath}
\usepackage{cite}
\usepackage[margin=0.75in]{geometry}

\begin{document}
\title{A Review of Energy Splitting of Alkali Atoms in a Magnetic Field}
\author{Karl Ahrendsen}
\maketitle{}

\section{Introduction}
In Atomic, Molecular,
and Optical Physics, a topic of interest is the collisions between
basic 
objects, such as atoms, molecules, and elementary particles.
Our research group is focused on the study of spin-polarized electrons. 
An electron beam can be easily created and collided
with objects of interest(CITATION OF ELECTRON GUN WORK),
but this is not the most simplified interaction
which can occur, as electrons also have spin, which can distinguish one 
electron from another. We are interested in obtaining a source which
outputs electrons of only one polarization, referred to as a spin
polarized electron source.

Our research group has created a source of spin polarized
electrons using spin-exchange interactions with the alkali metal
rubidium.(CITATIONS, CITATIONS)  Our current method of obtaining 
polarized electrons involves first polarizing a rubidium vapor
with circularly polarized light. Then an unpolarized electron beam
is passed along the same path of the light 
through the polarized vapor. Spin exchange interactions
transfer the spin of the rubidium vapor to the electrons, and the 
result is a polarized electron beam. This approach makes intuitive
sense.

Another research group, Swensens, obtains the polarized electrons
in a different manner~\cite{swenson}. They pump their alkali metal, 
potassium, with
\emph{linearly} polarized light, perpendicular to the direction
of propogation of the electrons. This also must be done in the
presence of a magnetic field. The result is the same, a collection
of spin polarized alkali metal atoms, which could be used to create
a spin polarized electron beam. The goal of this paper is to thoroughly
explain how both of these approaches acheive the same result, as well 
as to examine the feasibility of performing this experiment in our
lab group.

This goal will  be accomplished by first explaining the basics 
of atomic physics for pumping alkali atoms, including notation, energy
levels, angular momentum, and selection rules. After this we
will examine how our description of the energy changes in the
presence of a magnetic
field. Once these basics of pumping have been established we will
examine the specific case of transverse optical pumping. 

\section{Basics of Energy Levels of Alkali Atoms}
The process of pumping a gas is based upon the transition
of an atom from some ground state to an excited state. 
It is therefore very important that one has a firm 
understading of the quantum mechanics involved in this 
process. For this reason, I begin with a review of the energy
levels of an atom. For the sake of simplicity, I will complete
this discussion using
Hydrogen as the example atom, but the equations may be 
easily adjusted to be appropriate for an alkali atom.
These modifications will be discussed in Section \ref{angularMomentum}
	\subsection{No Magnetic Field}
		\subsubsection{Energy Levels}
		\subsubsection{Angular Momentum}
        \label{angularMomentum}
		\subsubsection{Selection Rules}

	\subsection{With Magnetic Field}
		\subsubsection{Energy Levels}
		\subsubsection{Coupling}
		\subsubsection{Selection Rules}

\section{Transverse Optical Pumping}

	\subsection{Required Setup}

	\subsection{Allowed Transitions}

	\subsection{Pumping}

\section{Conclusion}
END OF PAPER

the available values of the mixing element $\theta_{13}$.
This angle is important because it could help to explain
where all of the "missing" matter in the universe is.
While measuring the angle itself will not explain this,
it will help to pave the way towards other studies, such
as the mass of the neutrino, that could explain the imbalance.
In the context of Electroweak theory, this angle is important
because it is one of the fundamental variables that is needed
to completely describe the theory.

The mixing of neutrinos is described by a mixing 
matrix, which has the form:

\begin{equation}
	\ket{\nu_\alpha} = \sum_{i=1}^{3} U_{\alpha i} \ket{\nu_i}
\end{equation}

In this expression, the kets on the right hand side make up
the different mass eigenstates, while the kets on the left hand
side make up the different flavor states. Each of 
the elements of the $U$ matrix then represent the probability
that a given mass eigenstate will have a certain flavor after 
a specified amount of time. Because neutrinos are frequently
traveling at very close to light speeds, this specified amount
of time can be converted to a distance. The $U$ matrix can be 
broken into the product of three matrices, each one with a
different angle which describes the mixing between two of the
three flavors. When these matrices are multiplied together, 
the result is:

\begin{equation}
	U=
	\begin{pmatrix}
		c_{12}c_{13} & s_{12}c_{13} & s_{13}e^{-i\delta}\\
		-s_{12}c_{23} -c_{12}s_{23}s_{13}e^{i\delta} & c_{12}c_{23} -s_{12}s_{23}s_{13}e^{i\delta} & s_{23}c_{13}\\
		s_{12}s_{23} -c_{12}c_{23}s_{13}e^{i\delta} & -c_{12}s_{23} -s_{12}c_{23}s_{13}e^{i\delta} & c_{23}c_{13}\\
	\end{pmatrix}
\end{equation}

where $c_{ij}$ represents $\cos(\theta_{ij})$, for example. In
this representation, the relation between mass eigenstates and 
flavor eigenstates becomes clearer. Each row represents one 
of the flavors and each column represents the contribution 
of the mass eigenstate to that flavor. From this point, the 
probabilities of oscillations between flavors could in principle be 
calculated, but it is very advantageous to use
previously established experimental results to simplify the problem. 

From previous results it has become clear that one of the neutrino 
masses is vastly different from the others. When this is the case,
the number of parameters needed to describe the oscillations is 
reduced from six to three. Previously, two mass differences, three 
mixing angles and one CP-violating phase were needed. With a one large
mass difference, the other mass difference is assumed to be zero and 
the needed parameters are reduced to just the large mass difference 
and the two mixing angles. We refer to
this different mass as $m_3$ and our three variables of interest are
$\Delta m^2$, $\theta_{23}$, $\theta_{13}$.

With this simplification, the survival and transmission probabilities
for electron and muon neutrinos can be written. The transmission that
we are interested in is the muon neutrino to electron neutrino
(or vice-versa) which is given by:

\begin{equation}
	\begin{aligned}
		P(\nu_e \rightarrow \nu_\mu) = \sin[2](\theta_{23})\sin[2](2\theta_{13})
		\sin[2](\dfrac{1.27\Delta m^2L}{E})
	\end{aligned}
\end{equation}

where $L$ is the distance the neutrino has traveled from its production
point in the atmosphere, $E$ is the neutrino energy in GeV, and 
$\Delta m^2$ is measured in MeV. This transmission probability
assumes that the neutrino is traveling through a vacuum. If instead the
neutrino is traveling through matter, this transmission 
probability will be adjusted by the matter effect. The matter
effect arises because of the large density of electrons in matter,
which will cause an increase in the amount of electron neutrinos which
are scattered due to the charged current interaction. The result of the
matter effect on the transition is dividing the probability by a 
factor of 
\begin{equation}
\pqty{\cos(2\theta_{13}) - A_{CC}/\Delta m^2}^2 + \sin[2](2\theta_{13}) 
\end{equation}
where
\begin{equation}
		A_{CC} = 2\sqrt{2}G_F N_e p
\end{equation}
in this expression, $G_F$ is the Fermi Constant, $N_e$ is the 
electron densities in the medium, and $p$ is the neutrino momentum.

Since this term is in the quotient and involves a difference, there
will be a certain energy for which the transmission probability
greatly increases as the difference becomes very small. In this case,
that energy happens to be at 3-10 GeV. In the paper, Figure 1 shows
how muon neutrinos passing through the earth will show greatly increased
probabilities of transition because of this effect. These neutrinos
which are passing through the opposite side of the earth will be
the focus of the data collected. If the number of electron neutrinos
coming through the earth is higher than expected, this is evidence
for a non-zero value of $\theta_{13}$.

The data used to evaluate this experiment consists of two parts. 
The first part is data from actual atmospheric neutrinos that was 
recorded between 1996 and 2001. The second part is simulated data
from Monte Carlo events not including the matter effect. By
comparing these two parts, one can determine whether the matter
effect makes a significant impact on the oscillations. 

The actual data is not just all of the events that occurred in the
detector between 1996 and 2001. Since the signature of a significant 
impact will be an excess of electron neutrinos coming through the earth, 
the total pool of events is whittled down to a smaller sample that
contains more electron neutrino events. Since the sample size of 
our data will effect the weight of each event, this serves to increase
the significance of electron neutrino events. 

The detection of these events is worth noting because of the uniqueness
of the detector. Super-Kamiokande is a water Cherenkov detector used
primarily for studying neutrinos. Neutrino events occur when a highly
energetic neutrino interacts with an electron via the weak force,
accelerating the electron to faster than light speeds. Because the particle 
is moving faster than the speed of light through water, it emits
a cone of light centered on the path of the particle. The light emitted
from this cone is then gathered by a set of the thousands of PMT tubes 
that line the interior of the detector, facing inwards toward the water. 
If the detector notes that a large set of PMT tubes has been triggered
in a pattern that represents a Cherenkov cone, the event is recorded.
By examining the characteristics of the Cherenkov cone, many details 
about the incident particle can be identified, such as the type of particle,
the energy of the particle, and the direction from which it came.

The simulated data comes from 100 years of Monte Carlo equivalent data.
These simulations are performed without taking into account neutrino
oscillations. Next, this data is fit to the actual data by first
adjusting for oscillation probability then taking into account 45
systematic error sources. While attempting to fit the data, the 
most probable values of $\Delta m^2$, $\sin[2](\theta_{23})$, and
$\sin[2](\theta_{13})$ 
are obtained. These values were found to be $2.5 \times 10^{-3}$ eV$^2$,
0.5, and 0.0, respectively. Furthermore, the electron neutrino
events were examined for asymmetry, to which none was found. This 
seems to suggest that the value of $\theta_{13}$ may indeed be zero.

Besides providing the most probable values of different variables,
the chi-squared test also puts limits on the upper and lower bounds
within certain confidence levels. For $\sin[2](\theta_{13})$ the value
is constrained to be less than 0.14 in a region where $\sin[2](\theta_{23})$
is between 0.37 and 0.65.

These analysis were all done assuming a normal hierarchy. The normal 
hierarchy is identified by having the closely spaced masses being the
lighter masses. An inverted mass hierarchy is also possible, where the
closely spaced masses are the heavier two. If this is the case, the 
electron like events will be reduced because of the smaller cross 
section of anti-electron neutrinos. This in turn means a weaker
constraint on the value of $\theta_{13}$. The same analysis was done as 
before for the inverted mass hierarchy. The resulting values 
were $2.5 \times 10^{-3}$ eV$^2$, 0.525, and 0.00625. Which are
judged to be not significantly different than the normal
hierarchy and this experiment does not show that one model
has preference over the other. 

Combining the result of this paper with the exclusion limits
of the CHOOZ experiment allows one to, with 90\% confidence, 
assert that $\sin[2](\theta_{13})$ is between 0 and 0.06, which
is in agreement with other experiments attempting to measure
the same value. 

\bibliographystyle{plain}
\bibliography{bibPotassium}

\end{document}
